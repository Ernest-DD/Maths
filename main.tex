\documentclass[14pt,a4paper]{article}

\usepackage{fullpage} % Package to use full page
\usepackage{parskip} % Package to tweak paragraph skipping
\usepackage{tikz} % Package for drawing
\usepackage[framemethod=TikZ]{mdframed}
\usepackage{amsmath}
\usepackage{amsthm}
\usepackage{indentfirst}
\usepackage[mathscr]{eucal}
\usepackage{pict2e}
\usepackage{epic}
\usepackage{hyperref}
\usepackage{mathrsfs}       % Quelques symboles supplémentaires
\usepackage{amssymb}        % encore des symboles.
\usepackage{amsfonts}       % Des fontes, eg pour \mathbb.
\usepackage[english]{babel} % Pour la traduction française
\usepackage{url}            % Pour citer les adresses web
%\usepackage{geometry}       % Gérer correctement la taille
\usepackage{graphicx} % inclusion des graphiques
\usepackage{wrapfig}  % Dessins dans le texte.
\usepackage{numprint}       % Histoire que les chiffres soient bien
\usepackage[utf8]{inputenc} % Lui aussi
\usepackage{float}
\usepackage{xcolor}
\usepackage{afterpage}
%\definecolor{Tan}{RGB}{213, 183, 139}
%\pagecolor{Tan}
%\color{black}


\usepackage{latexsym}
\usepackage{amstext}
\usepackage{amsxtra}
\usepackage{amscd}
\usepackage{amsopn}
\usepackage{lmodern}
\usepackage[utf8]{inputenc}
\usepackage[T1]{fontenc}
\usepackage{slashed}
\usepackage{mathtools}
\usepackage{microtype}
\usepackage{mathdots}
\usepackage{mathrsfs}
\usepackage[affil-it]{authblk}

\usepackage[marginratio={4:6, 5:7}, textwidth= 365pt]{geometry}
%\usepackage{mathabx}
%\usepackage{mathspec}
%\defaultfontfeatures{Mapping=tex}
%\defaultfontfeatures{Numbers=Proportional, WordSpace =1.6}

%\usepackage{framed,color}
%\definecolor{shadecolor}{rgb}{1,0.5,0.3}

\usepackage{empheq}
\newcommand*\widefbox[1]{\fbox{\hspace{2em}#1\hspace{2em}}}
\newtheorem{theorem}{Theorem}[section]
\newtheorem{corollary}{Corollary}[theorem]
\newtheorem{lemma}[theorem]{Lemma}
\newtheorem{claim}{Claim}[theorem]
\DeclareMathOperator{\Rem}{Rem}
\theoremstyle{plain}

\newenvironment{code}{%
\begin{mdframed}[linecolor=Green,innerrightmargin=30pt,innerleftmargin=30pt,
backgroundcolor=Black!5,
skipabove=10pt,skipbelow=10pt,roundcorner=5pt,
splitbottomskip=6pt,splittopskip=12pt]
}{%
\end{mdframed}
}

    \title{Proof Of The Collatz Conjecture}
    \author{Dogba Djaze}
    \affil{\small{Aerospace Engineering Department, \\ University of Bristol}}
    \date{\today}

%\textwidth=300pt

\begin{document}

\maketitle

	The Collatz conjecture is a conjecture in mathematics that concerns a sequence defined as follows: For any positive integer, if even, the next term is one half of it. If the previous term is odd, the next term is 3 times the previous term plus 1. The conjecture is that no matter what starting value, the sequence will always reach 1. In modular arithmetic notation, the sequence is defined as follows:
	
	\[
	f(n) = 
	\begin{cases}
	\frac{n}{2} , \text{ if } n \equiv 0 \bmod[2]\\
	3n+1, \text{ if } n \equiv 1 \bmod[2]\\
	\end{cases}
	\]
	
    Let suppose that all odd positive integers less than $M$ satisfy the Collatz conjecture. Therefore, proving or denying the conjecture simply comes in proving that the sequence will always or not give an odd positive integer less than the starting value $M$. This paper used such approach to validate the Collatz's conjecture.\\
	
\begin{lemma}
    Assuming that proving the conjecture comes in showing that the sequence always gives a value less than its initial value, all odd natural numbers except one converge to one under the Collatz sequence.
\end{lemma}
	
\begin{proof}
    Let first define all positive old integers as, 
    %
    \begin{equation}
        M = 2^{s_{1}}.~3^{s_{2}}.~t +1 
    \end{equation}
    %
    where $s_{1} \in N^{*}$, $s_{2} \in N$, $t \in N $ and $6\nmid t$. Assuming that $s_{1} \geq 2$, let implement the Collatz sequence using $M$.
    \begin{align}
        E_{1} &= \frac{3M+1}{4}= 2^{s_{1}-2}.~3^{s_{2}+1}.~t + 1
        \\
        E_{2} &= \frac{3S_{1}+1}{4} = 4(2^{s_{1}-4}.~3^{s_{2}+2}.~t + 1) 
        \\
        E_{n} &= \frac{3S_{n-1}+1}{4} = 2^{s_{1}-2n}.~3^{s_{2}+n}.~t + 1
    \end{align}
    
    Let assume that $2 \mid s_{1}$. Thus, $ 2 \mid E_{n} \Leftrightarrow  s_{1} = 2n \Rightarrow s_{1} \in N ~/~ 2\mid s_{1} $. Hence, 
    \begin{equation}
        E_{\frac{s_{1}}{2}} = 3^{s_{2}+\frac{s_{1}}{2}}.~t + 1 
    \end{equation}
    
     Since for $s_{1}\geq 2$, $~2^{s_{1}} > 3^{\frac{s_{1}}{2}}$ then, 
    \begin{align}
        2^{s_{1}}~3^{s_{2}} &> 3^{s_{2}+\frac{s_{1}}{2}} \\
        2^{s_{1}}~3^{s_{2}}~t+1 &> 3^{s_{2}+\frac{s_{1}}{2}}~t+1 \\
        M &> E_{\frac{s_{1}}{2}}
    \end{align}
    Therefore, if $2\mid s_{1}$, the collatz sequence will always give an odd positive integer less than the starting value. In other words, $\forall s_{1} \in N^{*}$ such that $2\mid s_{1}$, $2^{s_{1}}.~3^{s_{2}}.~t+1$ is always convergent.\\ 
    
    Now, let assume that $2\nmid s_{1}$. Thus, $s_{1}-2n=1$, which gives
    \begin{equation}
        E_{\frac{s_{1}-1}{2}} = 2.~3^{s_{2}+\frac{s_{1}-1}{2}}.~t+1
    \end{equation}
    Since for $s_{1}\geq 2$, ~ $2^{s_{1}}> 2.3^{\frac{s_{1}-1}{2}}$ then,
    \begin{align}
        2^{s_{1}}3^{s_{2}} &> 2.3^{s_{2}+\frac{s_{1}-1}{2}} \\
        2^{s_{1}}3^{s_{2}}t+1 &> 2.3^{s_{2}+\frac{s_{1}-1}{2}}t+1\\
        M &> E_{\frac{s_{1}-1}{2}}
    \end{align}
    Therefore, if $2\nmid s_{1}$, the collatz sequence will always give an odd positive integer less than the starting value.Since $ M > E_{\frac{s_{1}-1}{2}}$ implies that $M$ converges to one, then $E_{\frac{s_{1}-1}{2}}$  must also be convergent to one. In other words, all odd positive integers of the form $2^{1}.~3^{m}.~t+1$ are always convergent under the Collatz sequence. Therefore, $\forall s_{1} \in N^{*}$ such that $2\nmid s_{1}$, $2^{s_{1}}.~3^{s_{2}}.~t+1$ is always convergent.\\
\end{proof}
\end{document}
